\begin{abstract}
Musculoskeletal simulations are useful in biomechanics to investigate the causes of movement disorder, to estimate non-measurable physiological quantities or to study the optimality of animal movement.
We introduce \bioptim, a Python interface for optimal control of musculoskeletal models. 
\bioptim uses the direct multiple shooting formulation and it is interfaced with several nonlinear solvers.
The software is intended to be computationally efficient and, at the same time, easily customizable by the user.
It allows to quickly define a variety of biomechanical problems such as motion tracking or prediction, electromyography-driven simulation, parameter optimization, real-time moving horizon estimation, multiphase problems, etc. 
To show \bioptim's abilities, six examples are developed in this paper.
First, a muscle activation-driven pointing task is performed. 
Next, a twisting somersault is obtained with a quaternion-based model.
Then, a position controller using external forces is presented.
Next, a multiphase muscle activation-driven walking cycle is optimized.
Then, a real-time moving horizon estimation of muscle forces is performed.
Finally, a multiphase torque-driven vertical jump motion is presented. 
\end{abstract}

\textbf{Keywords -- Biomechanics, Optimization, Software, Optimal control, Musculoskeletal}

