This example is presented to introduce \bioptim's ability to provide real-time estimation of biomechanical variables.
The goal was to perform a real-time estimation of dynamically consistent joint kinematics and muscle forces, using a moving horizon estimation (MHE) approach (i.e. an optimization approach that uses a series of measurements observed over time). 
A shoulder elevation motion was performed with a 4-DoF ($\bf{q}$) arm actuated by 19 Hill-type muscle elements.
The control inputs of the model were the muscle activations ($\bf{a}$).
The MHE implementation consists in splitting the OCP into a succession of smaller one for processing fixed-size subsets of the tracking data moving forward in time. 
Each time one subproblem is solved, a new measurement is added, the oldest one is discarded and a new subproblem is defined. 
Due to their similarities, the solution of the previous OCP is a good initial guess to the new one. 
The dynamical consistency of the final solution is enforced by continuity constraints on the initial state. 
Each objective function (Eq.~\ref{eq:ocp_exMHE}) was written as the sum of three terms: tracking reference joint angles ($\bf{q^*}$), states and muscle activations regularizations (i.e., least-square criteria): 
\\ 
\[ 
\resizebox{0.9\columnwidth}{!}{$ 
\begin{aligned}
\mathcal{J} = &\int_t^{t+t_{mhe}}\underbrace{\omega_1´(\|\boldsymbol{q} - \boldsymbol{q^*}\|^{2})}_{\mathtt{TRACK\_STATE}}~ 
+ ~ \underbrace{\omega_2\|\boldsymbol{q\|^2}}_{\mathtt{MIN\_STATE}} 
+ ~ \underbrace{\omega_3\|\boldsymbol{a\|^2}}_{\mathtt{MIN\_ACTIVATION}}~dt, 
\end{aligned}   
$}  
\addtag  
\label{eq:ocp_exMHE}  
\]  

\noindent where $\omega_1 =10^3$, $\omega_2 = 10$, $\omega_3 = 10^2$ and $t_{mhe}$ is duration of each sub-problem. 

In this example, reference data of an $\SI{8}{\second}$ series of four arm elevations were generated at $\SI{100}{\Hz}$, by computer simulation.
A centered Gaussian noise (mean = 0, std = $0.005\:q^*(t)$) was added to $q^*$, to simulate experimental-like joints angle measurements.
Using a windows size of 7 nodes (i.e., $\SI{210}{\milli\second}$), the estimator ran at about $\SI{33}{\Hz}$ (one in three reference data frame was sent to the estimator to simulate experimental-like conditions), i.e., two and half times faster than standard biofeedback ($\SI{13}{\Hz}$, \cite{kannape2013self}).
The MHE was able to forecast the movement kinematics with a root mean square error of $1.3\pm\SI{0.7}{\degree}$ while providing a realistic estimation of muscle forces close to the ground truth with a root mean square error of $11.1\pm\SI{14.9}{\newton}$ (Fig.~\ref{fig:MHE_results}).


