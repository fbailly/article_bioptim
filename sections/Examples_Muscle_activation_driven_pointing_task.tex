The goal was to achieve a muscle activation driven pointing task using a 2-DoF, 6-muscle arm model. 
In addition to muscle-induced torques, pure torques could compensate for the model weaknesses.

Term $\#1$ of the objective function (Tab.~\ref{tab:Muscle_activation_driven_pointing_task}) corresponds to the pointing tasks described by a Mayer term (heaviest weight), to superimpose two markers, the first one fixed in the ulna system of coordinates and the second one fixed in the scene.
Terms $\#2$ and $\#3$ were added for control regularization (muscle activation and torques) and $\#4$ for state regularization. 
The movement lasted for 2 seconds and was discretized using 51 shooting nodes.
%

%
\begin{table}[h!]
\caption{\small Objective terms of the activation-driven pointing task. The names of the functions correspond to the nomenclature used in \textit{Bioptim}.}
\label{tab:Muscle_activation_driven_pointing_task}
\centering
\begin{tabular}{c c c c}
\toprule 
& Type & Function & Weight \\ 
\midrule
$\#1$ & Mayer & ALIGN\_ MARKERS & $1e6$ \\ 
\midrule
$\#2$ & Lagrange & MINIMIZE\_ MUSCLE\_ CONTROL & $1e1$ \\ 
\midrule
$\#3$ & Lagrange & MINIMIZE\_ TORQUE & $1e1$ \\ 
\midrule
$\#4$ & Lagrange & MINIMIZE\_ STATE & $1e1$ \\
\bottomrule
\end{tabular}
\end{table}
%

%
The problem was solved using IPOPT and ACADOS resulting in two significantly different solutions.
ACADOS provided a 16 times smaller optimized cost (Tab.~\ref{tab:Perfs_and_detailed_implementations_of_each_example}), which illustrate the pitfalls of local minima as well as the benefits of having straightforward access to different solvers.  
Indeed, the ACADOS-based solution (Fig.~\ref{fig:snapshots_activation_driven_pointing}, top) makes good use of gravity to minimize the control inputs, while the IPOPT-based solution (Fig.~\ref{fig:snapshots_activation_driven_pointing}, bottom) moved the arm in the opposite direction and was stuck in a local minimum (still achieving the task though). 
It is worth mentioning that for the purpose of this illustration, no constraint was given about the shoulder range of motion to ensure physiological muscle trajectories. 

 
 

%
\begin{figure*}[t!]
\centering
\includegraphics[width=\textwidth]{figures/activation_pointing_snapshots_acados.png}\\
\vspace*{0.5em}
\includegraphics[width=\textwidth]{figures/activation_pointing_snapshots_ipopt.png}
\caption{Snapshots of an optimized muscle activation driven pointing task. Top: using ACADOS. Bottom: using IPOPT.}
\label{fig:snapshots_activation_driven_pointing}
\end{figure*}
%









